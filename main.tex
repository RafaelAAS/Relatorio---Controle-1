%%%%%%%%%%%%%%%%%%%%%%%%%%%%%%%%%%%%%%%%%%%%%%%%%%%%%%%%%%%
% --------------------------------------------------------
% Tau
% LaTeX Template
% Version 2.4.4 (28/02/2025)
%
% Author: 
% Guillermo Jimenez (memo.notess1@gmail.com)
% 
% License:
% Creative Commons CC BY 4.0
% --------------------------------------------------------
%%%%%%%%%%%%%%%%%%%%%%%%%%%%%%%%%%%%%%%%%%%%%%%%%%%%%%%%%%%

\documentclass[9pt,a4paper,twocolumn,twoside]{tau-class/tau}
\usepackage[portuguese]{babel}
%% Spanish babel recomendation
% \usepackage[spanish,es-nodecimaldot,es-noindentfirst]{babel} 
%% Draft watermark
% \usepackage{draftwatermark}
%----------------------------------------------------------
% TITLE
%----------------------------------------------------------

\journalname{Relatório de Sistema de Controle 1 - Engenharia de Computação}
\title{Controle de Posição em Sistema Massa-Mola com Conexão Flexível}

%----------------------------------------------------------
% AUTHORS, AFFILIATIONS AND PROFESSOR
%----------------------------------------------------------

\author[a,1]{Rafael Anacleto Alves de Souza}
\author[a,2]{Elvis Correia Lopes dos Santos}
\author[a,3]{Guilherme de Oliveira Costa}
\author[a,4]{Pedro de Carvalho Cedrim}


%----------------------------------------------------------

\affil[a]{Instituto de Computação, Universidade Federal de Alagoas – Campus A.C. Simões\\
\textsuperscript{1}\texttt{raas@ic.ufal.br}, 
\textsuperscript{2}\texttt{ecls@ic.ufal.br}, 
\textsuperscript{3}\texttt{goc@ic.ufal.br},
\textsuperscript{4}\texttt{pcc@ic.ufal.br}}

\professor{Prof. Dr. Icaro Bezerra Queiroz de Araujo}

%----------------------------------------------------------
% FOOTER INFORMATION
%----------------------------------------------------------

\institution{Instituto de Computação, Universidade Federal de Alagoas}
%\footinfo{Relatório elaborado com classe \LaTeX\ Tau}
\theday{setembro de 2025}
\leadauthor{Souza, Santos, Costa e Cedrim}
\course{Engenharia de Computação - Sistemas de Controle 1}

%----------------------------------------------------------
% ABSTRACT AND KEYWORDS
%----------------------------------------------------------

\begin{abstract}    
    O projeto tem como objetivo o estudo e implementação de um sistema de controle de posição em uma configuração massa-mola com duas massas acopladas por molas, representando situações práticas de sistemas com ligação elástica entre atuador e carga.
   
    O problema central consiste em manter a posição controlada da massa de saída, mesmo diante das oscilações e acoplamentos internos do sistema. Para isso, pretende-se desenvolver um modelo matemático, realizar simulações computacionais.

    O foco é a obtenção do modelo matemático que descreve a dinâmica do sistema e sua simulção em malha aberta, servindo de base para o projeto e implementação do controlador de posição.
\end{abstract}

%----------------------------------------------------------

\keywords{sistema massa-mola, controle de posição, acoplamento elástico, resposta dinâmica, two-mass system}

%----------------------------------------------------------

\begin{document}
		
    \maketitle 
    \thispagestyle{firststyle} 
    \tauabstract
    % \tableofcontents
    % \linenumbers 
    
%----------------------------------------------------------

\section{Modelagem Matemática}

Nesta seção, o sistema de duas massas é modelado matematicamente. A abordagem utilizada baseia-se na aplicação das leis de Newton e no uso de impedâncias mecânicas no domínio de Laplace para derivar as equações diferenciais que governam o comportamento dinâmico do sistema. O sistema físico sob análise é representado pela Figura 1.

%adicionar figura o digrama que está nas anotações do elvis
O modelo considera duas massas, $M_1$ e $M_2$, conectadas por uma mola de constante elástica $K_2$. A massa $M_1$ está conectada a um ponto fixo por uma mola de constante elástica $K_1$. Uma força externa Y(t) é aplicada à massa $M_1$, e a variável de saída de interesse é deslocamento da massa $M_2$, denotado por $X_2(t)$. Os coeficientes de atrito viscoso para cada massa são representados por $A_1$ e $A_2$.

Aplicando a Segunda Lei de Newton para cada massa e transformando as equações para o domínio de Laplaca, obtemos o seguinte sistema de equações em formato matricial:
\[
\begin{bmatrix}
   M_1 s^2 + A_1 s + K_1 + K_2 && -K_2 \\
   -K_2 && M_2 s^2 + A_2 s + K_2
\end{bmatrix}
\begin{bmatrix}
    X_1 (s) \\
    X_2 (s)
\end{bmatrix}
=
\begin{bmatrix}
    Y(s) \\
    0
\end{bmatrix}
\]

Para obter a função transferência $G(s) = \frac{X_2 (s)}{Y(s)}$, que relaciona o deslocamento da massa 2 com a força aplicada da massa 1, podemos utilizar a Regra de Cramer. O resultado é a seguinte função de transferência:
\begin{equation}
    G(s) = \frac{X_2 (s)}{Y(s)} = \frac{K_2}{(M_1 s^2 + A_1 s + K_1 + K_2)(M_2 s^2 + A_2 s + K_2) - K_2 ^2}
\label{eq: FT}
\end{equation}

Para uma análise inicial e simplificada do sistema, os coeficientes de atrito viscoso ($A_1$ e $A_2$) foram desconsiderados ($A_1 = 0$, $A_2 = 0$). Essa simplificação permite focar na dinâmica fundamental gerada pela interação das massas e molas. A função de transferência simplificada é:
\begin{equation}
    G(s) = \frac{K_2}{(M_1 s^2 + K_1 + K_2)(M_2 s^2 + K_2) - K_2 ^2}
\label{eq: FT_simp}
\end{equation}

Expandindo o denominador, obtemos:
\begin{equation}
    G(s) = \frac{K_2}{M_1 M_2 s^4 + (M_1 K_2 + M_2 K_1 + M_2 K_2)s^2 + K_1 K_2}
\label{eq: FT_exp}
\end{equation}

Utilizando os valores das massas medidas experimentalmente, $M_1 = 87 g = 0.087 kg$ e $M_2 = 80g = 0.080kg$, função de transferência com os parâmetros $K_1$ e $K_2$ a serem determinados é:
\begin{equation}
    G(s) = \frac{K_2}{0.00696 s^4 + (0.080 K_1 + 0.167 K_2)s^2 + K_1 K_2}
\label{eq: FT_valoresmassa}
\end{equation}

O diagrama de blocos do sistema em malha aberta é apresentado na Figura 2.
%Criar um diagrama de blocos simples com uma entrada Y(s), um bloco contendo G(s) e uma saída X2(s)

\section{Análise do Modelo}

A estabilidade do sistema em malha aberta é determinada pelos polos da função de transferência, que são as raizes do polinômio característico do denominador:
\begin{equation}
    \Delta (s) = 0.00696 s^4 + (0.080 K_1 + 0.167 K_2)s^2 + K_1 K_2 = 0
\label{eq:raizes}
\end{equation}

Como o polinômio não possui termos de ordem ímpar, os polos do sistema sem amortecimento estarão localizados no eixo imaginário. Isso indica que o sistema em malha abeta é marginalmente estável.

O comportamento dinâmico esperado para uma entrada degrau é uma resposta oscilatória sustentada, composta pela superposição de duas senoides com frequências distintas (modos de vibração), característica de sistemas de duas massas acopladas. A determinação exata dos pontos e das frequências de oscilações depende dos valores das constantes elásticas $K_1$ e $K_2$, que serão levantados experimentalmente na próxima etapa do projeto.

\section{Simulação Computacional}

Para validar o comportamento dinâmico previsto pelo modelo matemático, foi realiada uma simulação computacional utlizando o software MATLAB/Simulink.

Para a simulação, foram adotados valores preliminares para as constantes das molas, baseados em componentes comerciais comuns (ex: $K_1 = 100N/m$ e $K_2 = 150N/m$). É importante resaltar que estes valores serão substituídos pelos valores reais medidos.

%simulação do MATLAB, gerar gráfico da resposta ao degram e ao impulso e inseri-los
\subsection{Análise dos Resultados da Simulação}
O gráfico da resposta ao degrau (Figura X) mostra um comportamento puramente oscilatório e sem atenuação, como previsto pela análise teórica do modelo não amortecido. Observa-se a superposição de duas frequências de oscilações, confimando a natureza do sistema de segunda ordem com acoplamento. A ausência de amortecimento na simulação leva a uma oscilação perpétua, o que na prática será atenuado por atritos não modelados.

\section{Conclusão Parcial}

Nesta etapa, o modelo matemático para o sistema de duas massas e duas molas foi obtido com sucesso. A análise da função de transferência indicou que o sistema sem atrito é marginalmente estável e as simulações computacionais confirmaram o comportamento dinâmico oscilatório previsto.

Este modelo matemático em malha aberta servirá como base fundamental para a próxima etapa do projeto, que consistirá na montagem do protótipo físico e na validação experimental. Os dados coletados do sistema real serão comparados com os resultados da simulação para refinar o modelo, se necessário, antes de prosseguir para o projeto de controlador de posição.



\printbibliography

%----------------------------------------------------------

\end{document}
